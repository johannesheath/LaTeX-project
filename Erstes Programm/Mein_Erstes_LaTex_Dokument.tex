\documentclass[12pt, a4paper]{article} %Dokumentinformationen/Präambel
\usepackage{graphicx} %Paket für Bilder
\usepackage{caption} %Paket für Bildunterschriften
\usepackage[german]{babel} %Paket für Rechtschreibung
\usepackage{listings}
\usepackage{titlesec}
\usepackage{tikz}
\usepackage{float}
\usetikzlibrary{tikzmark}
\usetikzlibrary{matrix}
\usepackage{array}
\UseRawInputEncoding


% Für Quelltext
\usepackage{listings}
\usepackage{color}
\definecolor{mygreen}{rgb}{0,0.6,0}
\definecolor{mygray}{rgb}{0.5,0.5,0.5}
\definecolor{mymauve}{rgb}{0.58,0,0.82}
\definecolor{myrice}{RGB}{236,236,144}
\lstset{
  keywordstyle=\color{blue},commentstyle=\color{mygreen},
  stringstyle=\color{mymauve},rulecolor=\color{black},
  basicstyle=\footnotesize\ttfamily,numberstyle=\tiny\color{mygray},
  captionpos=b, % sets the caption-position to bottom
  keepspaces=true, % keeps spaces in text
  numbers=left, numbersep=5pt, showspaces=false,showstringspaces=true,
  showtabs=false, stepnumber=2, tabsize=2, title=\lstname}

\begin{document} %Beginn des eigentlichen Dokuments

\thispagestyle{empty}

%Beginn Titelseite
\begin{title}
	
	\centering{\scshape\Large Projektwoche 2022 am EMA \par}
	\vspace{5cm}
	
	{\scshape\Huge Mein erstes LaTex Dokument \par}

	\vspace{5cm}
	{\Large\itshape Hubert Jaxon \par} %Hier deinen Namen eintragen
	

\end{title}
%Ende Titelseite

\newpage

\tableofcontents %erzeugt Inhaltsverzeichnis automatisch

\newpage

\section{\"Uber mich}

%Hier folgt eine Möglichkeit, etwas in LaTex aufzulisten
\begin{description}
\item[Name:]
Hubert Jaxon %Hier eintragen
\item[Alter:] 
69
\item[Hobbys:]
Essen
\item[Lieblingsfach:]
Sachkunde
\item[Lieblingsessen:]
Quark
\item[Lieblingsmusik:]
Mongolischer Halsgesang
\item[Lieblingsquark:] %Eigenen Punkt erstellen
Vollfettquark
\end{description}

\section{Bild in LaTex}

%Lade im Internet ein Bild herunter, lade es in Overleaf hoch und binde es mit dem Befehl
%	\includegraphics[height=50mm]{Dateiname}
%in LaTex ein
%Mit dem Befehl
%	\caption{Bildunterschrift}
%kannst du eine Bildunterschrift hinzufügen, z.B. die Quelle

\begin{figure}[h]%h = here (t = top; b = bottom; p = page)
\centering\includegraphics[height=50mm]{jaxon.jpeg}\caption{Eine Weide}
\end{figure}

\section{Arten von Kapiteln}

Mit \verb|\section{Name}| beginnst du ein neues Kapitel, was automatisch im Inhaltsverzeichnis angezeigt wird, sofern der Befehl \verb|\tableofcontents| auf der entsprechenden Seite eingebunden ist.

\subsection{Unterkapitel}

Mit \verb|\subsection{Name}| f\"ugst du ein Unterkapitel ein.

\subsubsection{Unterunterkapitel}

Mit \verb|\subsubsection{Name}| f\"ugst du ein Unterunterkapitel ein.

\section{Textgr\"o\ss e}
\label{sec:Textgr\"o\ss e}

%Im Dokumentkopf (Zeile 1) legst du die Default-Schriftgröße des Dokuments fest (hier sind das 12pt). Wenn du z.B. auf der Titelseite andere Schriftgrößen verwenden möchtest, 
%helfen folgende Befehle.

%Du änderst die Schriftgröße mit
%	\Schriftgröße
% Die Namen dieser Größen stehen in der Tabelle

%Ersetze die 'x' in der Tabelle mit einem Wort in der entsprechenden Schriftart, welche links daneben steht


\begin{table}[h] 
\centering
\begin{tabular}{l | l}
\bf{Gr\"o\ss e} & \bf{Beispiel} \\
\hline 
Huge & \Huge Jaxon\\
\hline
huge & \huge Jaxon\\
\hline
LARGE & \LARGE Jaxon\\
\hline
Large & \Large Jaxon\\
\hline 
large & \large Jaxon\\
\hline
normalsize & \normalsize Jaxon\\
\hline
small & \small Jaxon\\
\hline
footnotesize & \footnotesize Jaxon\\
\hline
scriptsize & \scriptsize Jaxon\\
\hline
tiny & \tiny Jaxon\\
\end{tabular}
\caption{\label{tab4}Schriftgr\"o\ss en in LaTex }
\end{table}

\section{Textfarbe}

%Erkundige dich im Internet, wie man Farben in LaTex ändern kann.
%Markiere im folgenden Text beliebige Textpassagen rot, grün, cyan und gelb.

Lorem ipsum dolor sit \textcolor{blue}{amet}, consectetur adipiscing elit, sed do eiusmod tempor incididunt ut labore et dolore \fcolorbox{magenta}{white}{magna} aliqua. Ut enim ad minim veniam, quis \colorbox{myrice}{\textcolor{brown}{nostrud}} exercitation ullamco laboris nisi ut aliquip ex ea commodo consequat. Duis aute irure dolor in reprehenderit in voluptate velit esse cillum dolore eu fugiat nulla pariatur.

\section{Schriftart}

%Informiere dich auf der Webseite
%	https://de.overleaf.com/learn/latex/Font_sizes%2C_families%2C_and_styles#Font_families
%wie man in LaTex die Schriftart ändern kann

%Schreibe anschließend Passagen im folgenden Text mit folgenden Schriftarten: Roman (upright) serif , sans serif, typewriter (monospace)

Lorem ipsum dolor sit amet, \textsf{consectetur adipiscing elit,} \texttt{sed do eiusmod tempor incididunt ut labore et dolore magna aliqua.} \textrm{Ut enim ad minim veniam,} \textsf{quis nostrud exercitation ullamco laboris nisi ut aliquip ex ea commodo consequat.} \texttt{Duis aute irure dolor in reprehenderit in voluptate velit esse cillum dolore eu fugiat nulla pariatur.}


\section{Quellcode}

%Mit dem Befehl
%	\begin{lstlisting}[language=*Programmiersprache* ,caption=*Code Name*] ... \end{lstlisting}
%fügst du Quellcode in LaTex ein. Wie du siehst, können die Programmiersprache und der Text unter dem Code
%ausgewählt werden

%Füge in diesem Kapitel entweder einen Quellcode deiner Wahl ein, oder kopiere den Quellcode aus dem Java-Programm im GitHub Ordner:

\begin{lstlisting}[language=Java , caption=Kopfrechentrainer]
public class Kopfrechentrainer {
  public static void main (String[] arguments) {
    int x = Zufall.getInt();
    int y = Zufall.getInt();
    System.out.println(x + " + " + y);
    System.out.println("=");
    int z = Kon.readInt();
    int anzahl = 0;
    
    while (z == x + y) {
      anzahl = anzahl + 1;
      x = Zufall.getInt();
      y = Zufall.getInt();
      System.out.println(x + " + " + y);
      System.out.println("=");
      z = Kon.readInt();
    }
    
    if (z != x + y) {
      System.out.println("Da hast du einen Fehler gemacht!");
      System.out.println(x + " + " + y + " ergibt " + (x + y) + "(!)");
      System.out.println("Du hast insgesamt " + anzahl + " Aufgaben gelöst.");
    }
  }
}
\end{lstlisting}

\newpage
\section{Eigene Projekte}
\subsection{Tabellen}

\begin{table}[h]
    \centering
    \begin{tabular}{|c|c|c|}
    \hline 
    \multicolumn{3}{|c|}{Titel} \\
    \hline
    \multicolumn{2}{|c|}{test} & cell 5 \\
    \hline
    cell5 & 2test & cell7 \\
    \hline
    \end{tabular}
    \caption{Tafelreplika}
    \label{tab:tafelreplika}
\end{table}


An dieser Stelle referenziere ich \ref{tab:tafelreplika}, welche auch eine Tabelle ist.
An dieser zweiten Stelle referenziere ich \ref{sec:Textgr\"o\ss e}, welches ein Kapitel ist.


\subsection{Matheumgebungen}

An dieser Stelle beginnt eine Aufzählung von \texttt{mathematischen Gleichungen}:

\begin{enumerate}
    \item $a^2 + b^2 = c^2$
    \item $(a+b)\cdot(a-b)=a^2-b^2$
    \item $\frac{a}{b}:\frac{c}{d}=\frac{a}{b}\cdot\frac{d}{c}$
    \item $f(x)=ln(\frac{\frac{x}{2}}{\pi})+\sin{x}$
    \item $f_{a}(x)=e^{\frac{x^2}{x+2}+2x}+ax$
    \item $g\prime(x)=2x$
    \item $\int_{a}^{b} f(x)dx$
    \item $(\sum\nolimits_{i=1}^n f_{i}(x))^\prime=\sum\nolimits_{i=1}^n f^\prime_{i}(x)$
    \item \[ (\sum\nolimits_{i=1}^n f_{i}(x))^\prime=\sum\nolimits_{i=1}^n f^\prime_{i}(x) \] 
\end{enumerate}


\end{document}
